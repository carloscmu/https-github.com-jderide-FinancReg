\section{Fri, Feb 23rd}
\[r_i(x_i;F_\epsilon(p,\cN))\in\argmax_r\lset \Ex\{ \pi(r(1-\epsilon))\}\mset r\geq x_i,\,\Ex\{ \pi(r(1-\epsilon))\}\geq 0\rset\]
\subsection{Analyzing the shock}
The shock have two possible sources: Idiosyncratic $\epsilon^I$, and Non-Idiosyncratic (coming from the network), $\epsilon^{Nt}$.  For insititution $i$, the probability of receiving the idiosyncratic shock is given by $q_i$, and we assume that $\epsilon_i^I\sim {\rm U}[0,UB]$, and that 
\[\epsilon=\begin{cases}
\epsilon^I&\Pro\{\cdot\}=q_i\\
\epsilon^{Nt}&1-q_i
\end{cases}\]

We realize that the contagion mechanism and the shock propagation are going to be analyzed with different perspective
\begin{itemize}
\item The idiosyncratic risk can be initiated at node $i$ with probability $q_i$
\item Once the institution $i$ receives the shock, it propagates it with an intensity of $p$ times the shock \jd{note that here, we can consider $p<1$ for mitigation effect, or $p>1$ for an increasing effect}. 
\item The shock only propagates by simple paths between nodes (no revisiting allowed)

\begin{assumption}{\rm (shock propagation)}\label{assshprop}
The shock {\bf always propagates}, independently of the distress condition of the institution.
\end{assumption}
\item The final form for the shock faced by each institution is given by the equation
\begin{equation}\label{eqshockprop}
\epsilon = \left(\sum_{n=1}^{|N|-1} p^nA^n+I\right)q\epsilon^I
\end{equation}
\end{itemize}

\jd{Check the information available for each agent: Is the node totally visible for each agent?} We will continue assuming perfect information wrt the network

\subsection{$i$-maximization problem}
Define the {\emph modified agent maximization problem}
\begin{align*}
\max_r&\;\Ex\{ \pi(r(1-\epsilon))\}\\
\suchthat&\;r\geq x,\\
&\;\Ex\{ \pi(r(1-\epsilon))\}\geq 0,
\end{align*}
where $\epsilon$ comes from Equation~(\ref{eqshockprop}).  Note that by the linearity of the propagation, $\epsilon$\footnote{Simple paths?} followed the distribution of $\epsilon^I$, modified by a constant (easy to compute).  Thus, define these coefficients as $S_i$\footnote{$S=(\sum p^nA^n+I)q$}, and let's compute the expectation, 
\[\pi(\tau)=\begin{cases}a^0-a^1\tau& \tau\geq \lambda\\ 0&{\rm o.w.}\end{cases}\]
\begin{eqnarray*}
\Ex\{ \pi(r(1-\epsilon))\}&=&\Ex\lset\,\Ex\{ \pi(r(1-\epsilon))|r(1-\epsilon)\geq\lambda\}+\Ex\{ \pi(r(1-\epsilon))|r(1-\epsilon)<\lambda\}\rset\\
&=&\left(\int_{\{\epsilon:r(1-\epsilon)\geq \lambda\}}(a^0-a^1r(1-\tau))\Pro(d\tau) \right)\Pro\{r(1-\epsilon)\geq\lambda\}\\
&=&\begin{cases}
0&1-\frac{\lambda}{r}\leq 0\\
a^0-a^1r(1-\Ex\{\epsilon\})&\frac{1}{S\cdot UB}\left(1-\frac{\lambda}{r}\right)\geq 1\\
\left(1-\frac{\lambda}{r}\right)\int_0^{1-\frac{\lambda}{r}}(a^0-a^1r(1-t))\,\frac{dt}{S\cdot UB}&{\rm o.w.}
\end{cases}\\
&=&\begin{cases}
0&1-\frac{\lambda}{r}\leq 0\\
a^0-a^1r(1-\Ex\{\epsilon\})&\frac{1}{S\cdot UB}\left(1-\frac{\lambda}{r}\right)\geq 1\\
\frac{1}{S^2\cdot UB^2}\left(1-\frac{\lambda}{r}\right)^2(a^0-a^1\frac{r+\lambda}{2})\,&{\rm o.w.}
\end{cases}
\end{eqnarray*}