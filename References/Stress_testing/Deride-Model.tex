% July 28, 2015 (Tuesday, 17:32 PST) jd

%\documentclass[titlepage,draft,12pt]{article}
%\documentclass[titlepage,12pt,letterpaper]{article}
\documentclass[12pt,letterpaper]{article}
%\hoffset=-30pt
%\voffset=-30pt
%\textwidth=490pt
%\textheight=650pt
\input macawlrs.tex
\newcommand{\jd}[1]{\textcolor{blue}{${\textrm{Julio: }}${#1}}}
%% \renewcommand{\thefootnote}{\fnsymbol{footnote}}

\setlength{\topmargin}{-0.5in}
\setlength{\textheight}{9in}
\setlength{\oddsidemargin}{-.125in}
\setlength{\textwidth}{7in}

%\usepackage[urw-garamond]{mathdesign}

%\usepackage[T1]{fontenc}

\usepackage[cmintegrals,cmbraces]{newtxmath}
\usepackage{ebgaramond-maths}
\usepackage[T1]{fontenc}
\usepackage{svg}

%\usepackage[scale=0.845]{geometry}

\usepackage{fancyheadings}
\usepackage{lastpage}
\lhead{\tt Deride-Ram\'irez - Working note}
\chead{\tt SNL-FRB}
\rhead{\tt \today}
\cfoot{\tt \small Page \thepage\ of \pageref{LastPage}}

\pagestyle{fancyplain}


\usepackage{lineno}
\linenumbers

\title{A stochastic programming model for systemic financial resiliency}
\usepackage{authblk}
\author[1]{Julio Deride \thanks{jaderid@sandia.gov}}
\author[2]{Carlos Ram\'irez \thanks{carlos.ramirez@frb.gov}}
\affil[1]{Sandia National Laboratories, USA}
\affil[2]{Federal Reserve Board, USA}
\renewcommand\Authands{ and }

\begin{document}
\maketitle
\baselineskip=15pt

\section{Problem}
Consider a network $\rsG=(V,\,E)$ consisting on a set of $n$ nodes, $V=\{1,\ldots,n\}$, and a set of $m$ undirected edges $\{e_{ij}\}\in E$. Each node $i$ represents one of the institutions identity, and each edge $e_{ij}$ represents the \emph{correlation} or contagion factor between two entities. Let $(\Omega,\rsA,\Pro)$ be a probability space ...


Consider a two-stage model, where on the first stage there is a random shock happening on the nodes.   Let $\xi_i$ a Bernoulli random variable such that $\xi^0_i\in\{0,1\},\,i=1,\ldots,n$ represents the \emph{distress state} of node-$i$ on the network.  On the other hand, the second stage captures the behavior of the \emph{shock's propagation} over the network.  In order to define this, consider the stochastic process ${\bf P}$ modeling the probability of contagion of a node, given that one of its neighbor is distressed, i.e., if $\xi^1_i\in\{0,\,1\}$ represents the distress state of node $i$ in the second stage, then
\[{\bf P}_{ij}=\Pro\lset \xi^1_i(\cdot)=1\,\mset\,\xi^0_j=1\rset\,\quad e_{ij}\in E,\,\forall i,j\in V\]

The problem is now to minimize the total cost of the system under shocks on the network.  For this, the regulator is set to solve the problem of minimizing an overall cost, consisting on implementation cost and contagion cost, by deciding an optimal capital requirement.  Let $x^0$ be the decision policy, $x^0\in[0,1]^n$ such that $x^0_i$ represents the policy required at entity $i$, and ${\bf x}^1(\cdot)$ be a decision policy regarding the second stage (not sure if needed or not).  The optimization problem is given by
\[(\rsP)\quad \begin{array}{rl}\min_{\{x^0,{\bf x}^1(\cdot)\}}&\phi^0(x^0)+\Ex\lset \phi^1(\cdot,x^0,{\bf x}^1(\cdot))\rset\\
\suchthat&f^0(x^1)\leq 0 \\
&f^1(x^0,{\bf x}^1(\omega),\omega)\leq 0,\,\omega-\as\\
&x^0\in[0,1]^n,\,x^1:\Omega\to\reals^N\end{array},\]
where $\phi^0$ is the total cost of implementing a capital requirement policy, and $\phi^1$ is the total cost of the second stage (probably related to the contagion cost). Here, the network constraints are included in the constraints $\{f^0,f^1\}$ and for a random realization $\omega$ and a given policy $x^1$, the cost $\phi(\omega,x^0,{\bf x}^1(\omega))$ should reflect the cost of the contagion on the system. For example, one can be interested in minimizing the expected cost of the contagion, but it is easy to incorporate a risk-measure for minimizing, for example, a measure like $C-Var$ of the tail of the distribution of distressed nodes.

\section{Tue, Feb 20th}
We explore a model with the following features
\begin{itemize}
\item[1.] Consider a graph $G=(N,E)$, where each node represents a financial institution, and each edge reflects financial transactions between two institutions. 

\item[2.] Instituion $i$ faces a financial shock, represented as $\epsilon_i$, which impacts its assets over liabilities ratio, defined as $r_i=\frac{A_i}{L_i}$, \footnote{Capital?}.  Additionally, we consider that a financial institution is under \emph{distress} if its ratio is under a (given) threshold $\lambda\in(0,1)$.  Thus,
\[i\;{\rm under\,distress}\iff r_i(1-\epsilon_i)<\lambda\]

\item[3.] There is a central decision maker, focus on the stability of the system.  We discussed the information that is available to this regulator, and propose a mechanism to oversee the overall stability within the financial network thorugh a constraint over the ratio, given by $x_i$.

\item[4.]  The financial institutions decide their ratio by maximizing their profits\footnote{utility?}, given by a function $\pi_i$, with the minimum level of A/L ratio, i.e.,
\[r_i(x_i;p)\in\argmax_r\lset \Ex^p\{ \pi(r)\}\mset r\geq x_i,\,r\in R_i\rset\]
Additionally, assuming that the function $\pi_i$ is nonincreasing on $r$ (and no further restrictions are imposed), the individual solution to this problem is given by $r_i^*=x_i$, i.e., the financial institution sets its ratio at minimum possible level.

\item[5.] There is contagion on the network, described in its stationary state as follows: if institution $i$ gets distressed, there is a probability $p$ that it affects its immediate neighbor, $p^2$ by a 2-edge neighbors, and so on.  Defining the set $\{j\to i\}$ as the set of all possible simple paths coming to node $i$, and $d(j,i)$ the distance between $j$ and $i$ (amount of edges between them), the expected shock\jd{Assuming that there is no \emph{amplification} of shocks} is given by
\[\epsilon_i=\sum_{j\to i} p^{d(i,j)}\epsilon_j\]
and by defining the matrix $A_{ij}=\sum_{j\to i} p^{d(i,j)}$, the acceptable shocks are the solution of the eigen problem for the matrix $A$.  Moreover, we interpret $A$ as an stochastic (transition) matrix by enlarging it with an extra \emph{not distressed} node as follows,
\[\Tilde A=\left[\begin{array}{c|cccc|c}
&1&2&\ldots&n&ND\\
\hline
1&0&\sum_{2\to 1} p^{d(2,1)}&\ldots&\sum_{n\to 1} p^{d(n,1)}&1-\sum A_{1\cdot}\\
\vdots&\vdots&\ddots&\vdots&\vdots\\
n&\sum_{n\to 1} p^{d(n,1)}&\vdots&\ldots&0&1-\sum A_{1\cdot}\\
\hline
ND&0&0&0&0&1
\end{array}\right]\] 
This is a stochastic matrix, and thus, it has an eigenvalue with value 1, wich associated eigenvector $\tilde \epsilon^0$.  Let's consider the first $n$ components as acceptable shocks $\epsilon^0$ for the corresponding nodes.
\item Finally, consider the optimization problem solved by the central planner: set the ratio level\jd{sth about the condition previously stated}, such that it minimizes the total amount of financial institutions under distress.  Let $y_i\in\{0,1\}$ a binary variable such that $y_i=1$ if insititution $i$ is under distress or $y_i=0$ otherwise, and let $M>0$ large enough such that
\begin{align}\label{form1}
\min_{x,y}&\,\sum_{i=1}^n y_i+\phi(x,y)\\
\suchthat&r_i(x_i)(1-\epsilon^0_i)-\lambda\leq M(1-y_i),\,i=1,\ldots,N
\end{align}
where $\phi$ is a cost function associated to the policy $x$ and the instituions on distress.  Note that this formulation depends on $p$ and the topology of the network through the selection of the $\epsilon^0$.
\item The optimization problem \ref{form1} can have a robust formulation by considering an ambiguity set for the parameter $p$, thus
\begin{align}\label{formr1}
\min_{x,y}\sup_{p\in\cA(p_0)}&\,\sum_{i=1}^n y_i+\phi(x,y)\\
\suchthat&r_i(x_i)(1-\epsilon^0_i)-\lambda\leq M(1-y_i),\,i=1,\ldots,N
\end{align}
\item Finally, we are looking for a representative agent formulation of the benevolent social planner problem, such that the solutions of both problems coincide.  For example, one wild guess is to consider the formulation proposed in \cite{?} \jd{Citation needed}, where ambiguity is considered as a  family of possible models for the parameter $p$, along with a probability distribution over these models, $\alpha$.  Therefore, the central planner problem has the following form
\[\max_x\Lset \Ex^p\Lset\sum_i u_i(x_i)\Rset -\frac{\mu}{2}{\rm Var}^p\left(\sum_i u_i(x_i)\right)-\frac{\theta}{2}{\rm Var}_\alpha\Ex\left(\sum_i u_i(x_i)\right)\Rset\] 
\end{itemize}
\bibliographystyle{plain}
\bibliography{refer}


\end{document}
