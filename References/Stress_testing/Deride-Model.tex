% July 28, 2015 (Tuesday, 17:32 PST) jd

%\documentclass[titlepage,draft,12pt]{article}
%\documentclass[titlepage,12pt,letterpaper]{article}
\documentclass[12pt,letterpaper]{article}
%\hoffset=-30pt
%\voffset=-30pt
%\textwidth=490pt
%\textheight=650pt
\input macawlrs.tex
\newcommand{\jd}[1]{\textcolor{blue}{${\textrm{Julio: }}${#1}}}
%% \renewcommand{\thefootnote}{\fnsymbol{footnote}}

\setlength{\topmargin}{-0.5in}
\setlength{\textheight}{9in}
\setlength{\oddsidemargin}{-.125in}
\setlength{\textwidth}{7in}

%\usepackage[urw-garamond]{mathdesign}

%\usepackage[T1]{fontenc}

\usepackage[cmintegrals,cmbraces]{newtxmath}
\usepackage{ebgaramond-maths}
\usepackage[T1]{fontenc}
\usepackage{svg}

%\usepackage[scale=0.845]{geometry}

\usepackage{fancyheadings}
\usepackage{lastpage}
\lhead{\tt Deride-Ram\'irez - Working note}
\chead{\tt SNL-FRB}
\rhead{\tt \today}
\cfoot{\tt \small Page \thepage\ of \pageref{LastPage}}

\pagestyle{fancyplain}


\usepackage{lineno}
\linenumbers

\title{A stochastic programming model for systemic financial resiliency}
\usepackage{authblk}
\author[1]{Julio Deride \thanks{jaderid@sandia.gov}}
\author[2]{Carlos Ram\'irez \thanks{carlos.ramirez@frb.gov}}
\affil[1]{Sandia National Laboratories, USA}
\affil[2]{Federal Reserve Board, USA}
\renewcommand\Authands{ and }

\begin{document}
\maketitle
\baselineskip=15pt

\section{Problem}
Consider a network $\rsG=(V,\,E)$ consisting on a set of $n$ nodes, $V=\{1,\ldots,n\}$, and a set of $m$ undirected edges $\{e_{ij}\}\in E$. Each node $i$ represents one of the institutions identity, and each edge $e_{ij}$ represents the \emph{correlation} or contagion factor between two entities. Let $(\Omega,\rsA,\Pro)$ be a probability space ...


Consider a two-stage model, where on the first stage there is a random shock happening on the nodes.   Let $\xi_i$ a Bernoulli random variable such that $\xi^0_i\in\{0,1\},\,i=1,\ldots,n$ represents the \emph{distress state} of node-$i$ on the network.  On the other hand, the second stage captures the behavior of the \emph{shock's propagation} over the network.  In order to define this, consider the stochastic process ${\bf P}$ modeling the probability of contagion of a node, given that one of its neighbor is distressed, i.e., if $\xi^1_i\in\{0,\,1\}$ represents the distress state of node $i$ in the second stage, then
\[{\bf P}_{ij}=\Pro\lset \xi^1_i(\cdot)=1\,\mset\,\xi^0_j=1\rset\,\quad e_{ij}\in E,\,\forall i,j\in V\]

The problem is now to minimize the total cost of the system under shocks on the network.  For this, the regulator is set to solve the problem of minimizing an overall cost, consisting on implementation cost and contagion cost, by deciding an optimal capital requirement.  Let $x^0$ be the decision policy, $x^0\in[0,1]^n$ such that $x^0_i$ represents the policy required at entity $i$, and ${\bf x}^1(\cdot)$ be a decision policy regarding the second stage (not sure if needed or not).  The optimization problem is given by
\[(\rsP)\quad \begin{array}{rl}\min_{\{x^0,{\bf x}^1(\cdot)\}}&\phi^0(x^0)+\Ex\lset \phi^1(\cdot,x^0,{\bf x}^1(\cdot))\rset\\
\suchthat&f^0(x^1)\leq 0 \\
&f^1(x^0,{\bf x}^1(\omega),\omega)\leq 0,\,\omega-\as\\
&x^0\in[0,1]^n,\,x^1:\Omega\to\reals^N\end{array},\]
where $\phi^0$ is the total cost of implementing a capital requirement policy, and $\phi^1$ is the total cost of the second stage (probably related to the contagion cost). Here, the network constraints are included in the constraints $\{f^0,f^1\}$ and for a random realization $\omega$ and a given policy $x^1$, the cost $\phi(\omega,x^0,{\bf x}^1(\omega))$ should reflect the cost of the contagion on the system. For example, one can be interested in minimizing the expected cost of the contagion, but it is easy to incorporate a risk-measure for minimizing, for example, a measure like $C-Var$ of the tail of the distribution of distressed nodes.
\bibliographystyle{plain}
\bibliography{refer}


\end{document}
