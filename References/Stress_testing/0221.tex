\section{Wed, Feb 21st}
Continuing the idea of a (benevolent) central planner, let's define the following modified utility functions for each agents:
\[\tilde{\pi}_i=\begin{cases} \pi_i&{\rm institution}\,i\,{\rm operates\,normally}\\
0&{\rm institution\,is\,on\,distress}\end{cases}\]
Therefore, we expect that the Central Planner solves the following problem
\[\max_x\Lset \Ex^p\Lset\sum_i \tilde\pi_i(x_i)\Rset -\frac{\mu}{2}{\rm Var}^p\left(\sum_i \tilde\pi_i(x_i)\right)-\frac{\theta}{2}{\rm Var}_\alpha\Ex\left(\sum_i \tilde\pi_i(x_i)\right)\Rset\]

Additionally, we modify the assumptions over the spread of the distress condition

\begin{assumption}{\rm (initial shocks)}\label{assinit}
Initial shock only affects one institution (node)
\end{assumption} 

\begin{assumption}{\rm (propagation)}\label{assprop}
Propagation occurs over simple paths
\end{assumption} 

Under Assumptions~(\ref{assinit},\ref{assprop}), one can formulates the probaility of distress of node $k$, $\Pro\{D_k\}$, by a simple recursion.  If $j$ and $i$ are directly connected, the probability is given by
\begin{eqnarray*}
\Pro_{j|i}&=&\Pro\{D_j|D_i\}\\
&=&\Pro\{r_j(1-\epsilon_j)<\lambda|D_i\}\\
&=&\Pro\{r_j(1-p\epsilon_i)<\lambda|D_i\}\\
&=&1-F_{\epsilon_i}\left(\frac{1}{p}\left(1-\frac{\lambda}{r_j}\right)\right)
\end{eqnarray*}
where $F_{\epsilon_i}$ corresponds to the cdf of $\epsilon_i$.  Finally, for every institution (node) of the network, the contagion will depend only on all the possible simple paths connecting the corresponding node and the initially infested \jd{?}. Therefore,
\begin{equation}
\Pro\{D_k\}=\sum_{l\to k} \Pro\{D_k|D_l\}\cdot\Pro\{D_l\}
\end{equation}

\subsection{Revisiting the institutions' problem}
Let's consider an stochastic problem, where each node $i$ faces uncertainty on the final ratio.  Denote as $\epsilon$ the random shock that the agent expects ($\Ex \epsilon=\bar\epsilon>0$), and assume that every agent is risk neutral \jd{focused on network effects} and their profits are homogeneous and linear: $\pi(r)=a^0-a^1\,r$, $a^0,a^1>0$. Thus, the agent maximization problem is given by
\[r_i(x_i;\bar \epsilon)\in\argmax_r\lset \Ex\{ \pi(r(1-\epsilon))\}\mset r\geq x_i,\,r\in R_i\rset\]

Additionally, we include explicitly the participation constraint, where agent-$i$ only participates in the economy if its expected profits are nonnegative.  
\[r_i(x_i;\bar \epsilon)\in\argmax_r\lset a^0-a^1(1-\bar\epsilon)r\mset r\geq x_i,\,r\in R_i,\,a^0-a^1(1-\bar\epsilon)r\geq 0\rset\]

Note that in this formulation we consider agents that proceed in a \emph{na\"ive} fashion by only maximizing their profits, without considering its exposure within the network.

Following steps: Solve the problem considering
\begin{enumerate}
\item Risk neutral CP
\item Risk averse and Ambiguity neutral
\item Risk averse and Ambiguity averse
\item Numerical example
\end{enumerate}

